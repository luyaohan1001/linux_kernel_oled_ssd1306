Source code hierarchy\+: \begin{DoxyVerb}            oled_sysfs
                |
             graphics
                |
             datalink
                |
          driver  oled.dts
\end{DoxyVerb}


Tested on Linux raspberrypi 5.\+10.\+103-\/v7l+ \#1529 S\+MP Tue Mar 8 12\+:24\+:00 G\+MT 2022 armv7l G\+N\+U/\+Linux (Raspberry Pi Buster.)

P\+DF documents generated (by doxygen) at /docs/latex/refman.pdf

\paragraph*{Demo\+: Displaying text and the dinosaur from chrome browser.}



\paragraph*{Documentation.}



\paragraph*{To compile.}

\begin{DoxyVerb}Install the kernel headers.
$ sudo apt install raspberrypi-kernel-headers

Compile
$ sudo make

Successful compile message example:
pi@raspberrypi:~/Projects/raspberrypi-4b/drivers/oled $ make
    make -C /usr/src/linux-headers-5.10.103-v7l+ \
            ARCH=arm CROSS_COMPILE=arm-linux-gnueabihf- \
            M=/home/pi/Projects/raspberrypi-4b/drivers/oled modules
    make[1]: Entering directory '/usr/src/linux-headers-5.10.103-v7l+'
      CC [M]  /home/pi/Projects/raspberrypi-4b/drivers/oled/oled_sysfs.o
      LD [M]  /home/pi/Projects/raspberrypi-4b/drivers/oled/oled_driver.o
      MODPOST /home/pi/Projects/raspberrypi-4b/drivers/oled/Module.symvers
      CC [M]  /home/pi/Projects/raspberrypi-4b/drivers/oled/oled_driver.mod.o
      LD [M]  /home/pi/Projects/raspberrypi-4b/drivers/oled/oled_driver.ko
    make[1]: Leaving directory '/usr/src/linux-headers-5.10.103-v7l+'
\end{DoxyVerb}


\paragraph*{To run\+:}

\begin{DoxyVerb}1. First apply device tree overlay by

    $ sudo make dtoverlay

2. Insert the kernel module

    $ sudo make insmod
\end{DoxyVerb}


\paragraph*{To check for printk log\+:}

\begin{DoxyVerb}    $ dmesg
\end{DoxyVerb}


\paragraph*{To remove the kernel module\+:}

\begin{DoxyVerb}    $ sudo dmesg
\end{DoxyVerb}


\paragraph*{To generate docs by doxygen}

\begin{DoxyVerb}    $ make doxygen

    $ cd /docs/html
\end{DoxyVerb}


\paragraph*{Kanban -\/ T\+O\+DO}


\begin{DoxyItemize}
\item \mbox{[}x\mbox{]} release-\/00\+: Minimal-\/viable kernel i2c bus module and simple configruation + fill-\/screen.
\begin{DoxyItemize}
\item Constructing Makefile, setup build-\/environment (linux kernel headers)
\item Understanding struct i2c\+\_\+client , struct i2c\+\_\+driver.
\item Implementing probe and remove callbacks when the kernel inserts/remove the driver.
\end{DoxyItemize}
\item \mbox{[}x\mbox{]} release-\/01\+: Add font / image support to the screen datalink layer.
\begin{DoxyItemize}
\item Reading and coding various display functionalities according to S\+S\+D1306 I2C interface defined by Solomon Systech datasheet.
\end{DoxyItemize}
\item \mbox{[}x\mbox{]} release-\/02\+: Add user-\/space interface through sysfs.
\begin{DoxyItemize}
\item Understanding struct kobject, kobj\+\_\+attrbute.
\item Providing implementation on the creation of the oled device as a sysfs folder.
\item Providing implementation on the creation of oled attributes such as display\+\_\+text, brightness, etc. as files in that sysfs folder.
\end{DoxyItemize}
\item \mbox{[} \mbox{]} release-\/03\+: Develop the dinosaur game on this screen.
\begin{DoxyItemize}
\item Add multi-\/threading protection to critical sections.
\item Develop user-\/space dinosaur game, interacting with the kernel module through oled\+\_\+sysfs.
\end{DoxyItemize}
\item \mbox{[} \mbox{]} release-\/04\+: Unit testing.
\begin{DoxyItemize}
\item T\+BD 
\end{DoxyItemize}
\end{DoxyItemize}